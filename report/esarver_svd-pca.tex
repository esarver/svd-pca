\documentclass[conference]{IEEEtran}
\IEEEoverridecommandlockouts
% The preceding line is only needed to identify funding in the first footnote. If that is unneeded, please comment it out.
\usepackage{cite}
\usepackage{amsmath,amssymb,amsfonts}
\usepackage{algorithm}
\usepackage{algorithmicx}
\usepackage{algpseudocode}
\usepackage{tikz}
\usetikzlibrary{arrows.meta, positioning}
\usepackage{listings}
\usepackage{graphicx}
\usepackage{textcomp}
\usepackage{xcolor}

\graphicspath{{../images/}}
\def\BibTeX{{\rm B\kern-.05em{\sc i\kern-.025em b}\kern-.08em
    T\kern-.1667em\lower.7ex\hbox{E}\kern-.125emX}}
\begin{document}

\title{Singular Value Decomposition and Principle Component Analysis}


\author{\IEEEauthorblockN{Edwin Sarver}}

\maketitle

%TODO: https://tex.stackexchange.com/questions/13675/use-graphviz-within-tex

\section{Introduction}
Singular Value Decomposition (SVD) and Principle Component Analysis (PCA) are 
mathematical algorithms that allow, primarily, for the simplification of data.
This project implemented a command-line application to compress an image by 
using SVD. PCA was used to analyze a %TODO dataset. 

\section{Algorithms}\label{algo}
Two main algorithms were used in this project: Singular Value Decomposition (SVD)
and Principle Component Analysis (PCA).

\subsection{Singular Value Decomposition (SVD)}
Singular Value Decomposition is a mathematical method that uses Linear Algebra
techniques to separate a matrix into the parts that are most mathematically 
important. A given matrix $$M$$ can be separated into three matrices $$U$$, 
$$\Sigma$$, and $$V$$.

%TODO describe U, Sigma, and V and what they mean.

\subsubsection{Theoretical Performance}
The purpose of the SVD algorithm in this project was to produce a compressed image. 
The implementation did not need to be fast. The performance of the algorithm was, 
therefore, determined by the space-efficiency of the file that was output. 

%TODO add Theoretical analysis of space efficiency

%TODO Theoretical error analysis

\subsubsection{Experimental Performance}

%TODO add Experimental analysis of space efficiency

%TODO Experimental error analysis

\subsection{Principle Component Analysis (PCA)}
Principle Component Analysis simplifies the analysis of multi-dimensional datasets by 
reducing the dimensionality of the data. This make data visualization easier by finding
which dimensions of the data have the greatest impact on the categorization of the data. 


\begin{figure}
	\centering
	\includegraphics[width=3.25in]{Edmonds-Karp_Sparse.png}
	\caption{Edmonds-Karp Performance on Sparse Graph: Run Time vs $VE^2$}
	\label{ff_perf_sparse}
\end{figure}


\section{Insights}
%TODO Efficiency, modularity, generalizablility, stability, etc. 
The modularity and generalizability of SVD and PCA make them valuable for 
many applications. Static image compression is just one of the many possible
ways that SVD can be utilized. Another application is for video compression by
taking the difference between frames and then compressing the resulting difference.
Any application that requires the compression or analysis of highly-dimensional data
can be a possible application for SVD and, in the case of analysis, PCA. 


\section{Test Cases}
In this project, many forms of testing were utilized to test the image processing implementation: Unit tests, correctness tests,
and performance tests.

\subsection{Unit Tests}
Unit tests were used to ensure that each part of the program functioned as intended even after
changes were made to the code base. Included in these tests were tests to verify the algorithms
described in section \ref{algo}. The PGM images used in the unit tests were very small and were
only large enough to ensure that the basic functionality of the SVD algorithm was correct.  


\subsection{Correctness Tests}
Correctness tests for the SVD implementation were manual because they required the analysis of images.
The image used to verify the correctness of the algorithms was the provided image of the College of 
Arts and Sciences building on the University of Akron campus. In the course of testing, several issues
were identified and corrected. Most of the issues that were identified arose from incorrectly calculating
the position in the matrix that a particular element in the compressed, binary image file should occupy. 

\subsection{Performance Tests}\label{perf_explanation} %TODO revisit this
Performance test were also manual. A range of images were downloaded from %TODO pgm site.
The images were translated into a binary implementation, then the original pgm was split 
into a header and data file with a python script. The split files were then compressed 
using the SVD compression algorithm given some applicable ranks. The resulting binary 
files were then compared against the binary files that were generated directly from the 
original PGM files for size. The compressed files were then uncompressed to pgm files 
and were inspected for visual fidelity and the error from the original image was 
calculated. The relative error for each rank on each image could therefore be calculated.

The relative error can also be calculated from the singular values in the $$\Sigma$$ matrix.
The program was therefore modified to print the singular values to a separate file for further
analysis.

%\subsection{Design}
% TODO
%\subsection{Results}
% TODO

\section{Conclusion}
Singular Value Decomposition is a highly generalizable, powerful algorithm that creates a simplified 
mathematical representation of a dataset that describes which components are most important to the 
overall representation of the dataset. This allows for image compression by removing less important 
components of the image, but also allows for data analysis by the same mechanism when used in Principle
Component Analysis.  


\begin{thebibliography}{00}
    \bibitem{b1} T. H. Cormen, C. E. Leiserson, R. L. Rivest, and C. Stein, Introduction to Algorithms, 3rd ed., The MIT Press, 2009, pp. 594--602, 709--731 
\end{thebibliography}

\end{document}
